\documentclass{article}

\usepackage{geometry}
\geometry{legalpaper, margin=1in}

\usepackage{fancyvrb}

\newcommand\iggy{\texttt{iggy}}

\title{ \iggy  \\
User's guide \\
version 0.6
}

\author{Sven Thiele}

\date{}


\begin{document}

\maketitle


\section{What is \iggy?}

\iggy\ is a tool for consistency based analysis of influence graphs and observed systems behavior.
For many biological systems knowledge bases are available that describe the interaction of its components 
 usually in terms of causal networks and influence graphs. 
In particular signed influence graphs where edges indicate either positive or negative effect of one node upon another.
Building upon a notion of consistency between biochemical/genetic regulations and high-throughput profiles of cell activity
\iggy\ implements methods to check the consistency of large-scale data sets and
 provides explanations for inconsistencies. 
In practice, this can be used to identify unreliable data or to indicate missing reactions. 
Further, \iggy\ addresses the problem of repairing networks and corresponding yet often discrepant measurements
in order to re-establish their mutual consistency and predict unobserved variations even under inconsistency
 
 
\section{Installation}

You can install the \iggy\ package by running:

\begin{Verbatim}[frame=single]
$ pip install --user iggy
\end{Verbatim} 


On Linux the executable scripts can then be found in \texttt{~/.local/bin}

and on Mac OS the scripts are under \texttt{/Users/YOURUSERNAME/Library/Python/2.7/bin}.



\section{Usage}

Typical usage is:
\begin{Verbatim}[frame=single]
$ iggy.py network.sif observation.obs --show_labelings 10 --show_predictions
\end{Verbatim}
For more options you can ask for help as follows:
\begin{Verbatim}[frame=single]
$ iggy.py -h        
usage: iggy.py [-h] [--no_zero_constraints]
           [--propagate_unambigious_influences] [--no_founded_constraint]
           [--autoinputs] [--scenfit] [--show_labelings SHOW_LABELINGS]
           [--show_predictions]
           networkfile observationfile

positional arguments:
  networkfile           influence graph in SIF format
  observationfile       observations in bioquali format

optional arguments:
  -h, --help            show this help message and exit
  --no_zero_constraints
            turn constraints on zero variations OFF, default is ON
  --propagate_unambigious_influences
            turn constraints ON that if all predecessor of a node
            have the same influence this must have an effect,
            default is ON
  --no_founded_constraint
            turn constraints OFF that every variation must be
            explained by an input, default is ON
  --autoinputs          compute possible inputs of the network (nodes with
            indegree 0)
  --scenfit             compute scenfit of the data, default is mcos
  --show_labelings SHOW_LABELINGS
            number of labelings to print, default is OFF, 0=all
  --show_predictions    show predictions
\end{Verbatim} 


\section{Input}

\iggy\ works with two kinds of data. 
The first is network data representing and influence graph model. 
The second is the experimental data, representing experimental condition and observed behavior.


\subsection{Network data}

The network data is represented as file in simple interaction format \texttt{SIF} as shown below.
Lines in the \texttt{SIF} file specify a source node, a relationship type (or edge type), and one target node.
For our influence graph models we have the edge types 
\texttt{1} for \emph{increases} 
and \texttt{-1} for \emph{decreases}.
The first line in the example below therefore states that
 \texttt{depor} has an \emph{activating} (\texttt{1}) influence on \texttt{shp2\_ph}.
The last line states that  \texttt{shp2\_ph} has a \emph{inhibiting} (\texttt{-1}) influence on \texttt{plcg}.
Duplicate entries are ignored. 
Multiple edges between the same nodes must have different edge types. 
Other edge types than \texttt{1} and \texttt{-1} will lead to a parsing error.

\begin{Verbatim}[frame=single,numbers=left]
depor_p       1         shp2_ph
gab1_mem_p    1         shp2_ph
shp2_ph       -1        pi3k
gab1_mem_p    1         pi3k
pi3k          1         gab1_mem_p
shp2_ph       -1        plcg
\end{Verbatim} 


\subsection{Experimental data}

The experimental data is given in the file format shown below.
Nodes which are perturbed in the experimental condition are denoted as \texttt{input}.
The first line of the example below states that \texttt{depor} has been perturbed in the experiment.
This means \texttt{depor} has been under the control of the experimentalist 
 and its behavior must therefore not be explained.
The behavior of a node can be either \texttt{+}, \texttt{-}, \texttt{0}, \texttt{notPlus}, \texttt{notMinus}.
Line 2 states that an \emph{increase} (\texttt{+}) was obeserved in \texttt{depor},
 as it is declared an \texttt{input} this behavior has been caused by the experimentalist.
Line 3 states that \texttt{stat5ab\_py} has \emph{decreased} (\texttt{-}) and
line 4 states that \texttt{ras} has \emph{not changed} (\texttt{0}).
Line 5 states that an \emph{uncertain decrease} (\texttt{notPlus}) has been observed in \texttt{plcg} and
line 6 states that an \emph{uncertain increase} (\texttt{notMinus}) has been observed in \texttt{mtorc1}.

\begin{Verbatim}[frame=single,numbers=left]
depor         = input 
depor         = +
stat5ab_py    = -
ras           = 0
plcg          = notPlus
mtorc1        = notMinus
\end{Verbatim}  


\section{Output}

\iggy\ produces presents the results of its analysis as text output. 
The output of \iggy\ can be redirected into a file using the \texttt{>} operator.
For example to write the results shown below into the file \texttt{myfile.txt} type:

\begin{Verbatim}[frame=single]
$ iggy.py network.sif observation.obs --show_labelings 10 --show_predictions > myfile.txt
\end{Verbatim}

In the following we will dissect the output generated by \iggy.
The first 3 lines of the output state the constraints that have been used to analyze network and data.
For our example it is the default setting with the following constraints.
For a deeper understanding of these constraints see~\ref{our_paper}

\begin{Verbatim}[frame=single,numbers=left]
 all observed changes must be explained by an predecessor
 no-change observations must be explained
 all observed changes must be explained by an input
\end{Verbatim}

Next follow some statistics on the input data. 
Line 4-5 tells us that the influence graph model given as \texttt{network.sif} 
consists of \texttt{96} nodes, 
with \texttt{116} edges with activating influence 
and \texttt{16} edges with inhibiting influence
and \texttt{0} edges with \texttt{Dual} or ambiguous influence.
%
Line 9 tells that the experimental data given as \texttt{observation.obs} in itself is \texttt{consistent},
which means it does not contain contradictory observations.
Line 11 tells that the experimental conditions consists of \texttt{14} perturbations marked as \texttt{input} nodes, 
that \texttt{12} nodes were observed as increased \texttt{+}, 
\texttt{10} nodes \emph{decreased} (\texttt{-}),
\texttt{20} nodes did \emph{not change} (\texttt{0})
\texttt{5} nodes were observed with an \emph{uncertain decrease} (\texttt{notPlus}),
\texttt{4} nodes were observed with an \emph{uncertain increase} (\texttt{notMinus}),
\texttt{74} nodes were \texttt{unobserved} and the experimental data contained \texttt{0} observations of things that are not in the given influence graph model.

\begin{Verbatim}[frame=single,numbers=left,firstnumber=4] 
Reading network network.sif ... done.
   Nodes: 96  Activations: 116  Inhibitions: 16  Dual: 0

Reading observations observation.obs ... done.

Checking observations observation.obs ... done.
   Observations are consistent.
   inputs: 14  observed +: 12  observed -: 10  observed 0: 20  observed notPlus: 5  
   observed notMinus: 4 unobserved: 74  not in model: 0
\end{Verbatim}

Then follow the results of the consistency analysis.
Line 14 tells us that network and data are inconsistent 
and that the size of a \emph{minimal correction set} (\texttt{mcos}) is \texttt{1}.
This means that at least \texttt{1} influence needs to be added to restore consistency.
For a deeper understanding of mcos see~\ref{mcos}.
Further the output contains at most \texttt{10} consistent labeling including correction set. 
This is because we choose to set the flag \texttt{--show\_labelings 10}.
In our example we have only \texttt{2} possible labelings. 
Each labeling represents a consistent behavior of the model (given mcos the corrections).
\texttt{Labeling 1},
tells it is possible that 
\texttt{STAT3\_n} and \texttt{PAK1} \emph{increase} (\texttt{+})
\texttt{IGF1\_act} does \emph{not change} (\texttt{0}) and that
\texttt{KS6A5/KS6A4} and \texttt{TNR1A/TNR1B} \emph{decrease} (\texttt{-}).
Line 26 tells us that this is a consistent behavior if \texttt{MTOR} would receive a positive influence, 
which is currently not included in the model.
\texttt{Labeling 2}, represents an alternative behavior,
 here  \texttt{PAK1} and \texttt{KS6A5/KS6A4} do \emph{not change} (\texttt{0}).
Please note that in this example both labelings are consistent under the same correction set.
In other example more than one minimal correction set can exists.

\begin{Verbatim}[frame=single,numbers=left,firstnumber=13]  
Computing mcos of network and data ... done.
   The network and data are inconsistent: mcos = 1.
  
Compute mcos labelings ... done.
Labeling 1:
   gen("STAT3_n") = +
   gen("PAK1") = +
   gen("IGF1_act") = 0
   gen("KS6A5/KS6A4") = -
   gen("TNR1A/TNR1B") = -

   labeled +: 2  labeled -: 2  labeled 0: 1 

   new_influence("observation.obs",gen("MTOR"),1)
   
Labeling 2:
   gen("STAT3_n") = +
   gen("PAK1") = 0
   gen("IGF1_act") = 0
   gen("KS6A5/KS6A4") = 0
   gen("TNR1A/TNR1B") = -

   labeled +: 1  labeled -: 1  labeled 0: 3 
   
   new_influence("observation.obs",gen("MTOR"),1)
\end{Verbatim}

Finally the prediction results are listed.
A prediction is a statement that hold under all labeling under all minimal repairs. 
For a formal definition of predictions see~\ref{our_paper}.
Here the predictions say that 
\texttt{STAT3\_n} \emph{always increases} (\texttt{+}),
\texttt{PAK1} \emph{newer decreases} (\texttt{NOT -}),
\texttt{IGF1\_act} always stays \emph{unchanged} (\texttt{0}),
\texttt{KS6A5/KS6A4} \emph{newer increases} (\texttt{NOT +}), and that
\texttt{TNR1A/TNR1B} \emph{always decreases} (\texttt{-}).

\begin{Verbatim}[frame=single,numbers=left,firstnumber=38]  
Compute predictions under mcos ... done.
   gen("STAT3_n") = +
   gen("PAK1") = NOT -
   gen("IGF1_act") = 0
   gen("KS6A5/KS6A4") = NOT +
   gen("TNR1A/TNR1B") = -
 
   predicted +: 1  predicted -: 1  predicted 0: 1  predicted NOT +: 1  
   predicted NOT -: 1  predicted CHANGE: 0
\end{Verbatim}  
 
\end{document}
